\documentclass{article}

\begin{document}

\title{Annotated Bibliography}
\author{Andrew Gleeson}

\maketitle


%\begin{thebibliography}{9}

%\bibitem{tanton}
\makebox[1 cm][l]{\textbf{\Large 1.}} Tanton, James. ``A Dozen Questions about Triangular Numbers.'' Math Horizons 13 (2005).
\begin{quote}
I'm using this to get new questions about properties about triangular numbers. For example, he proposes that no triangular number ends in 2, 4, 7, or 9, and that all even perfect numbers are triangular numbers.
\end{quote}
\vspace{5 mm}

\makebox[1 cm][l]{\textbf{\Large 2.}} Bruckman, Paul, Joseph B. Dence, Thomas P. Dence, and Justin Young. ``Series of Reciprocal Triangular Numbers.'' The College Mathematics Journal 44.3 (2013): 177-84. \\
\begin{quote}
This paper has some interesting but more advanced math about infinite series of reciprocal triangular numbers, which I hope to be able to understand enough to use.
\end{quote}\vspace{5 mm}

\makebox[1 cm][l]{\textbf{\Large 3.}} Fang, Jin-Hui. "Nonexistence of a Geometrical Progression That Contains Four Triangular Numbers." Electronic Journal of Combinatorial Number Theory 7 (2007).
\begin{quote}
Fang shows that no four triangular numbers are part of a geometric progression.
\end{quote}\vspace{5 mm}

\makebox[1 cm][l]{\textbf{\Large 4.}} Ulas, Maciej. "On Certain Diophantine Equations Related to Triangular and Tetrahedral Numbers." {\tt arXiv:0811.2477v1 [math.NT]}
\begin{quote}
A good resource for various theorems related to the relationships between triangular and tetrahedral numbers.
\end{quote}\vspace{5 mm}

\newpage

\makebox[1 cm][l]{\textbf{\Large 5.}} Nelsen, Roger B. Proofs without Words: Exercises in Visual Thinking. Washington, D.C.: Mathematical Association of America, 1993.
\begin{quote}
An extremely accessible resource for elementary visual proofs of triangular numbers. It also contains references for the specific proofs, but I'm only including this book in the annotated bibliography.
\end{quote}\vspace{5 mm}

\makebox[1 cm][l]{\textbf{\Large 6.}} OEIS Foundation Inc. (2011), The On-Line Encyclopedia of Integer Sequences, http://oeis.org/A000217.
\begin{quote}
Lots of information of where triangular numbers crop up in other areas. I intend to look into how often relatively simplex numbers show up, as per Sloane's gap.
\end{quote}\vspace{5 mm}

\makebox[1 cm][l]{\textbf{\Large 7.}} OEIS Foundation Inc. (2011), The On-Line Encyclopedia of Integer Sequences, http://oeis.org/A000292.
\begin{quote}
The same as the above, but for tetrahedral numbers.
\end{quote}
\end{document}