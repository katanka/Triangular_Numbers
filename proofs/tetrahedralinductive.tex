\documentclass{article}
\usepackage{amsmath}
\usepackage{amsthm}
\usepackage{amssymb}

\usepackage{lipsum}
\usepackage{fancyhdr}
\pagestyle{fancy}

%\newtheorem{tetrahedral}{Printed output}

\rhead{Andrew Gleeson\\
\today}
\begin{document}


\subsection*{Inductive proof of the tetrahedral number formula}
A tetrahedral number of size $n$ is the sum of the triangular numbers from 1 to $n$.

\subsubsection*{Givens}

\begin{enumerate}
	\item Let $g(n)$ be the $n$th triangular number.

		$g(n) = \frac{n(n+1)}{2}$
	\item Let $F(n)$ be the $n$th tetrahedral number.
	
		$F(n) = g(1) + g(2) + g(3) + \cdots + g(n)$
	\item Assume:
	
		$F(n) = \frac{n(n+1)(n+2)}{6}$
		
	\item Let $F(n)$ be the $n$th tetrahedral number.
	
		$F(n) = g(1) + g(2) + g(3) + \cdots + g(n)$
	

\end{enumerate}

\subsubsection*{Base Case}

\begin{enumerate}
	
	\item $F(1) = \frac{1(1+1)(1+2)}{6} = 1$
	\item $g(1) = \frac{1(1+1)}{2} = 1$
	\item $F(1) = g(1) = 1$

\end{enumerate}

\subsubsection*{Inductive Step}

\begin{enumerate}
	
	\item $F(n + 1) =  g(1) + g(2) + g(3) + \cdots + g(n) + g(n+1)$
	\item $F(n + 1) =  F(n) + g(n+1)$
	\item $F(n + 1) =   \frac{n(n+1)(n+2)}{6} + \frac{(n+1)(n+2)}{2}$
	\item $F(n + 1) =   \frac{n(n+1)(n+2)}{6} + \frac{3(n+1)(n+2)}{6}$
	\item $F(n + 1) =   \frac{n(n+1)(n+2) + 3(n+1)(n+2)}{6}$
	\item $F(n + 1) =   \frac{(n+1)(n+2) (n+3)}{6}$

\end{enumerate}

\begin{center}$\therefore F(n) = \frac{n(n+1)(n+2)}{6}$ \end{center}

\end{document}