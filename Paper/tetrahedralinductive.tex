\subsection*{Inductive proof of the tetrahedral number formula}

Let $F(n)$ be the $n$th tetrahedral number, and $g(n)$ be the $n$th trianhular
number. By definition, $F(n) = g(1)+g(2)+g(3)+\ldots+g(n)$

\subsubsection*{Claim}
\begin{align*}
F(n) = \frac{n(n+1)(n+2)}{6}
\end{align*}

\subsubsection*{Proof: Induction on $n$}
\quad\textbf{n=1}
\begin{align*}
F(1)=g(1)=1\\
F(1)=\frac{(1)(2)(3)}{6}=1
\end{align*}
\quad\textbf{Inductive Step}\\

Assume $F(k) = \frac{k(k+1)(k+2)}{6}$. We want to show that $F(k+1) =
\frac{(k+1)(k+2)(k+3)}{6}$
\begin{align*}
F(k + 1) =  g(1) + g(2) + g(3) + \cdots + g(k) + g(k+1)\\
	 F(k + 1) =  F(k) + g(k+1)\\
	 F(k + 1) =   \frac{k(k+1)(k+2)}{6} + \frac{(k+1)(k+2)}{2}\\
	 F(k + 1) =   \frac{k(k+1)(k+2)}{6} + \frac{3(k+1)(k+2)}{6}\\
	 F(k + 1) =   \frac{k(k+1)(k+2) + 3(k+1)(k+2)}{6}\\
	 F(k + 1) =   \frac{(k+1)(k+2) (k+3)}{6}
\end{align*}

\begin{center}$\therefore F(n) = \frac{n(n+1)(n+2)}{6}$ \end{center}